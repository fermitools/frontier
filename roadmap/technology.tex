
\chapter{Technology}

\section{Overview}

\begin{fixme}
This entire section is notes---not final text.
\end{fixme}

The internet is filled with widely accepted, production quality,
frameworks and tools used for delivering database, and other
information using the HTTP protocol.There are several general
technologies used for serving dynamic content pages including Tux,
Apache, PHP, Perl, Python, and Server-Side Java. Database access and
pool management for the database connections has been developed for
many of the server products as well, we know that we will need to
access oth Oracle, MySQL, and possibly other database
platforms. Caching of dynamic content pages is not normal, but it is
possible to build it into the server products, and also there are many
caching proxy server products which offer reasonable performance and
versitile configuration options to meet our needs.  In addition,
whatever products are deployed the server performance monitored. Some
of the products reviewed have monitoring features built in, and some
do not. There are many tools available for watching the overall
performance of machines running the servers such as memory, cpu, and
netword activity.

{reference L. Titchkosky, M. Arlitt, C. Williamson, ``A Performance
Comparison of Dynamic Web Technology,'' Performance Evaluation Review,
volume 31 number 3, December 2003, pp2-11.}

There is a broad array of server software approaches used on the
internet to deliver dynamic content pages including Tux, Apache, PHP,
Server-sise Java, Perl and python.

{reference
http://www.redhat.com/docs/manuals/tux/TUX-2.2-Manual/intro.html} Tux,
also known as Red Hat Content Accelerator, is a high performance
kernal-based web server for Linux. Red Hat Content Accelerator runs
partly within a custom version of kernel 2.4.x or higher and partly as
a user-space daemon. With a capable network card, Red Hat Content
Accelerator enables direct scatter-gather DMA from the page cache
directly to the network, thus avoiding data copies. Whenever Red Hat
Content Accelerator is unsure how to process a request or receives a
request it is unable to handle, it always redirects the request to the
user-space web server daemon to handle it in an RFC-compliant
manner. An example of this user-space web server daemon is Apache.It
is currently limited to serving static webpages and coordinating with
kernel-space modules, user-space modules, and regular user-space Web
server daemons to provide dynamic content. Regular user-space Web
servers do not need to be altered in any way for Red Hat Content
Accelerator to coordinate with them. However, user-space code has to
use a new interface based on the tux(2) system call. Red Hat Content
Accelerator also has the ability to cache dynamic content. Red Hat
Content Accelerator modules (which can be build in kernel space or in
user space; user space is recommended) can create "objects" which are
stored using the page cache. To respond to a request for dynamic data,
a Red Hat Content Accelerator module can send a mix of
dynamically-generated data and cached pre-generated objects, taking
maximal advantage of Red Hat Content Accelerator's zero-copy
architecture.


Apache can be used to serve dynamic content pages by including plugins
which are available for many languages including PHP, perl, python,
and Java. It has the advantage that most of the worlds web servers run
Apache, and it is well documented, highly configurable, and there are
many monitoring packages available for it.  PHP is a scripting
language specifically designed for the Web, and it is generally used
as a plug-in wiht Apache.

There are a number of Java products which provide server-side dynamic
content processing using modules, servlets, written in Java. Tomcat is
a servlet container that provides the official reference
implementation for both Java Servlets and Java Server
Pages. (http://jakarta.apache.org/tomcat) Jetty ia a Web server and
Java Servlet container written in Java. It is advertized as one of the
fastest servlet servers available. (http://jetty.mortbay.org/jetty)
Resin is a commercial Web server and Java servlet container freely
available to non-commercial users.

\hyphenation{language update}

\begin{table}[h]
\centering
\caption{What is this table about?}
\begin{tabularx}{\textwidth}{XXXXXX}\toprule
Product  & Performance & Configurability & Servlet Language & Servlet Update & Caching and Proxy Servers \\ \midrule
         &             &                 &                  &                &                            \\ \bottomrule
\end{tabularx}
\end{table}

\section{Caching and Proxy servers}

Caching is configurable for, or can be built into, many of the servers
discussed above. Another approach which can be used in conjunction
with the dynamic content servers is an independent layer of caching
provided by proxy servers. For a comprehensive overview of this
technology see (Duane Wessels, Web Caching) This is a well understood
technology used on the internet to provide performance and reliability
across the internet, although it is rarely used in conjunction with
dynamic content pages. Proxy caching servers provide many features
which make them attractive including authentication, request
filtering, response filtering, prefetching, translation and
transcoding, and traffic shaping. They can be interconnected in
various configurations of meshes, clusters, hierarchies.

There are many products available which are interesting for our
use. Squid is an open source package that runs on a wide range of unix
platforms and it is highly configurable. Netscape Proxy Server runs
oon unix systems as well and windoew NT, and is one of the first proxy
products. CacheFlow provides intellegent prefetching and refreshing
features. InfoLiberanis designed for reliability and fault
tolerance. There are several other products which can be purchased or
run on proprietary hardware.

Caching proxies provide the ability to configure the cache management
and cache sharing among groupings of servers. Cache management can be
controlled with several policies, like LRU. A variety of inter cache
protocols are available inclueing Internet Cache Protocal (ICP), Cache
Array Routing Protocal (CARP), Hypertext Caching Protocol (HTCP), and
Cache Digests.

System performance and reliability need to be planned and there are
several approaches to load balancing and fail over that can be
considered, if expensive hardware is not an option. High availability
systems can be built with, UPS, dual redundant power supplies,
processors, et cetera, and raid disks. Alternatively, some simple DNS
tricks and complex load balancing. Layer foru switches can be
configured to intellegently manage failure detection as well.  Load
sharing techniques include DNS round Robin, Layer four switches, CARP,
ICP.

Because of the popularity of this technology, there are numerous
monitoring tools availabel as well. Some Proxy Caching servers provide
SNMP interfaces for which there are numerous analysis and charting
utilities available. Also, log analysers are availale for logs
formatted in ``standard'' formats, which most PCS's provide. Commonly
used monitoring tools such as MRTG and RRDTool enable easy access to
the performance data coming from each server.

In additon to monitoring the software server itself, it is generally
useful to monitor machine performance. Tools such as Ganglia, nagios,
and tcpdump provide convenient solutions for monitoring distributed
systems.

Sun provides the base classes for JDBC, on which third party database
venders layer their specific implementations. Oracle and MySQL supply
jar files with their particular derived subclasses.

Some vendor database connections are lab wide limited by license. We
therefore consider database connections to be a critical resource
which must be monitored.  A pool of available database connections
will be provided.  This pool will be configurable as to the maximum
and minimum number of connections allowed and how long an idle
connection may remain open.  Server connection requests will be filled
from and returend to this pool.  Any available connection will be
given to the request.  If none is available a new connection will be
created up to the maximum pool size.  After that requests will be
queued until an existing connection is returned to the pool.

We belive it is possible to supply a connection pool with third party
software such as the DbConnectionBroker provided by JavaExchange.com
or the connection pool which comes with Tomcat.

Need to figure out JNDI

Discussion of the commercial products goes here: Java, servlets,
MySQL, JDBC, squid, ganglia, nagios, tcpdump, cURL, nameservice,
redundancy, \etc%Note that 'etc' carries a period already.


\section{Choices}

\begin{fixme}
What goes here?
\end{fixme}


